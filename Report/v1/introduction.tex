\chapter{Introduction}
\label{ch:intro}
\begin{center}
\textquote{\textit{Performance capture is a technology, not a genre; it's just another way of recording an actor's performance}.} - Andy Serkis
\end{center}

A talented painter or sculptor is able to imagine and reproduce the subtle details of a human face. Hours of training and endless manual adjustments are required before an arbitrary shape resembles a facial expression. Furthermore, a human eye is trained to notice subtle changes in the expression of another human being, which also applies to animated humans, thus even the smallest discrepancies in an animated model are easily detected. Though artists are able to produce good quality and appealing results, the limitations in budget and time so prominent in the entertainment industry motivate the development of more automated, faster and cheaper models.

The naive approach that allows for faster performance is the keyframe animation; though it is based on manual input, keyframe animation requires fewer frames as the intermediate expressions are interpolated over. However, this approach is still very laborious and time consuming. With the emergence of sophisticated systems and software for motion capture, development of methods for performance-driven animation has received significant attention from both the industry as well as the academia.
