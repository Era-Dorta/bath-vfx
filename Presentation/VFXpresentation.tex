\documentclass{beamer}
% \usepackage{lmodern}
%=====================================================================
% Color definition
\definecolor{jvagreen}{RGB}{0,104,	139}
\definecolor{jvagold}{RGB}{255, 255, 255}
\setbeamercolor{section in head/foot}{fg = jvagold, bg = jvagreen}
\usepackage{graphicx}
\usepackage{mathtools}
\usepackage{picture}
\usepackage{amsmath}
\usepackage{multimedia}
\DeclareMathOperator{\Tr}{Tr}
%\graphicspath{{../}}
\setbeameroption{show notes}
\usepackage{multicol}
\usepackage{caption}
\usepackage{textpos}

\usebackgroundtemplate
{
    \includegraphics[width=\paperwidth,height=\paperheight]{img/blank.jpg}%
}
\beamertemplatenavigationsymbolsempty

%=====================================================================
% Templates - headline, frametitle

\makeatletter
% Komprimiert die miniframe Kreise auf eine Linie
\beamer@compresstrue
\makeatother

% Definiert die headline
\setbeamertemplate{headline}
{ 
%\includegraphics[width=\paperwidth]{pic} % test logo
\begin{beamercolorbox}[wd=\paperwidth,right]{section in head/foot}
    \rule{\paperwidth}{1pt}
    %Vertikaler Abstand
    \vskip20pt
    %Fügt die Standard-Navi ein (miniframes)
    %\insertnavigation{\paperwidth}
    \vskip8pt
    %\rule{\paperwidth}{0.5pt}
    %\vskip25.5pt % same height of the example provided, but IMHO is too much
    \rule{\paperwidth}{1pt}
\end{beamercolorbox} 
}
\setbeamertemplate{footline}
{ 
%\includegraphics[width=\paperwidth]{pic} % test logo
\begin{beamercolorbox}[wd=\paperwidth,right]{section in head/foot}
    \rule{\paperwidth}{1pt}
    %Vertikaler Abstand
    \vskip10pt
    %Fügt die Standard-Navi ein (miniframes)
    \insertnavigation{\paperwidth}
    \vskip8pt
    \rule{\paperwidth}{0.5pt}
    %\vskip25.5pt % same height of the example provided, but IMHO is too much
    %\rule{\paperwidth}{1pt}
\end{beamercolorbox} 
}

% definition of the frametitle
\setbeamertemplate{frametitle}
{
\vskip-24pt % to shift up the frametitle
\hbox{ 
 \begin{beamercolorbox}[wd=.0675\textwidth]{} % left shift
 \end{beamercolorbox} 
 \begin{beamercolorbox}[sep=4pt]{section in head/foot}
 \insertframetitle
 \end{beamercolorbox} 
 }
}

\mode<presentation>
{
    \setbeamertemplate{itemize item}[circle]
    \setbeamercolor{itemize item}{fg = jvagreen}
    \setbeamertemplate{itemize subitem}[circle]
    \setbeamercolor{itemize subitem}{fg = jvagreen}
}

\renewcommand\footnoterule{}

\AtBeginSection[]
{
  \begin{frame}<beamer>
    \frametitle{Outline}
    \tableofcontents[currentsection]
  \end{frame}
}

%\logo{\includegraphics[height=0.6cm]{img/cde_basic_green}}
%\let\oldequation=\equation
%\let\endoldequation=\endequation
%\renewenvironment{equation}{\vspace{1cm}\begin{oldequation}}{\end{oldequation}\vspace{15mm}}

\begin{document}

\addtobeamertemplate{frametitle}{}{%
\begin{textblock*}{100mm}(-0.9cm,-0.6cm)
\includegraphics[height=0.5cm,width=1.5cm]{img/uob-logo-white-transparent}
\end{textblock*}
\begin{textblock*}{100mm}(0.97\textwidth,-0.6cm)
\includegraphics[height=0.5cm,width=1cm]{img/cde_tag_white}
\end{textblock*}
}

\title[Performance-driven Facial Animation]{
  Performance-driven Facial Animation}

% Optional: a subtitle to be dispalyed on the title slide
% \subtitle{Show where you're from}
% \subtitle{Presented by: Ieva Kazlauskaite}
% The author(s) of the presentation:
%  - again first a short version to be displayed at the bottom;
%  - next the full list of authors, which may include contact information;
\author[Garoe Dorta Perez, Ieva Kazlauskaite, Richard Shaw]{
   Garoe Dorta Perez, Ieva Kazlauskaite, Richard Shaw } 


% The institute:
%  - to start the name of the university as displayed on the top of each slide
%    this can be adjusted such that you can also create a Dutch version
%  - next the institute information as displayed on the title slide
\institute[University of Bath]{
University of Bath \\
Centre For Digital Entertainment
}

% Add a date and possibly the name of the event to the slides
%  - again first a short version to be shown at the bottom of each slide
%  - second the full date and event name for the title slide

\date{27 May 2015}

% TITLE PAGE
\begin{frame}[plain]
  \titlepage
\end{frame}

%\section{Overview}
% CONTENT PAGE
%\begin{frame}
%  \frametitle{Overview}
%  \tableofcontents
%\end{frame}

%----------------------------------------------------------------------
\section{Richard's Part}
\begin{frame}{Frame name}

\end{frame}

\begin{frame}{Frame name}

\end{frame}

%----------------------------------------------------------------------
\section{Ieva's Part}
\begin{frame}{Frame name}

\end{frame}


\begin{frame}{Frame name}

\end{frame}


%----------------------------------------------------------------------
\section{Garoe's Part}
\begin{frame}{Frame name}

\end{frame}


\begin{frame}{Frame name}

\end{frame}


%----------------------------------------------------------------------
\section{Results}
\begin{frame}{Frame name}
%\begin{center}
%\begin{figure}
%\movie[width=0.45\textwidth, autostart, loop]
%        {\includegraphics[width=0.25\textwidth]{img/arkanoid}}{img/arkanoid1.mov}
%~
%%,showcontrols
%\caption*{\tiny{Left shows rigid body physics, right shows crossplatform interaction.}}
%\end{figure}
%\end{center}
\end{frame}


\begin{frame}{Frame name}

\end{frame}



%\begin{center}
%\begin{figure}
%\includegraphics[width=0.5\textwidth]{img/6phongReflexion} ~
%\includegraphics[width=0.5\textwidth]{img/8phongShading}
%\caption*{\tiny{Left shows flat shading, right shows Gouraud shading.}}
%\end{figure}
%\end{center}
%
%\begin{itemize}
%\setlength\itemsep{0.5em}
%\item 
%\end{itemize}

%\begin{multicols}{2}
%\begin{figure}[b!]
%\includegraphics[width=0.4\textwidth]{img/cloth_directions}
%\end{figure}
%
%\vfill
%\columnbreak
%\vspace*{\fill}
%\small{where $F$ are Fresnel terms, $\eta$ are Fresnel coefficients, $g$ is a Gaussian lobe, $k_d$ is a scattering constant, $\gamma$ are Gaussian widths, $\theta_h = (\theta_i+\theta_r)/2$ and $\phi_d = \phi_i-\phi_r$. }
%\end{multicols}


%----------------------------------------------------------------------

\end{document}
